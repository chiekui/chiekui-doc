\documentclass[10pt,a4paper]{jarticle}
\usepackage[top=30truemm,bottom=30truemm,left=20truemm,right=20truemm]{geometry}
\usepackage{amsmath}
\usepackage[]{multicol}
\title{智慧喰 白書(draft)}
\author{水樹素子 sonoko\_mizuki}
\begin{document}
\maketitle
\begin{abstract}
この世には、様々な分散台帳システムが数多に存在しているが、未だ研究の余地があり課題を抱えている。
よってここに、人類を一歩すすめるため新しい分散台帳システム"智慧喰"を提案する。
この文書は、"智慧喰"の思想、設計、プロトコル、そして未来についてまとめたものである。
\end{abstract}

\begin{multicols}{2}

\section{Introduction}
\subsubsection{由来と目的}
このシステムの名前を”智慧喰”としたのは以下の目的を満たすことが最終的な到達点だからである。
\begin{itemize}
  \item 智慧を食べるものとして知識を持った人々が集い、叡智を組み合わせより強固なシステムと成り、人類の礎を築く事を目的とする。
  \item 情報を食べるものとして異なる台帳が集い、巨大な1つの台帳と成り、利用者は情報の位置を知らなくとも参照、変更可能なシステムを築く事を目的とする。
\end{itemize}

\subsubsection{なせ作るか?}

\section{智慧喰い}
ここで私の提案するシステム、”智慧喰”について説明する。
\subsubsection{定義}
以下の条件を満たすものを{\bf 智慧喰システム}と定義する。
\begin{enumerate}
  \item Commandを受け取り、システムの情報を変更する事ができる。
  \item Queryを受け取り、システムの状態を返す事ができる。
  \item 後述するデータモデルを持つ。
  \item システムを構成するPeerを任意に増減することができる。
\end{enumerate}
以下の条件を満たすものを{\bf 智慧喰システムの完全Peer}と定義する。
\begin{enumerate}
  \item 以下のEndpointを持ち、Peer間、Client-Peer間で通信を行う事ができる。
    \begin{itemize}
      \item API: Command/Query用Endpoint
      \item Consensus: 合意形成用Endpoint
      \item Internal: 内部通信用Endpoint
      \item Supervise: 監視用Endpoint
    \end{itemize}
  \item 指定された形式でデータを保存することができる
  \item 合意形成に参加し、データの整合性を確認することができる
\end{enumerate}
上記の条件から3項を抜いたものを {\bf 智慧喰システムの準Peer}と定義する。

\subsubsection{構成要素}
智慧喰システムの完全Peerは以下の装置によって構成されている。なお存在理由を右に記述する。
\begin{enumerate}
  \item Command/Queryを受け取り、受け取った事実をトリガーとして任意のイベントを発生させる装置。受け取った事実を記録するため
  \item 受け取ったCommand/Queryの受容可能かを判断する装置。正しいProtocolに基づいた物のみを選別するため
  \item Commandを下記のデータフローに基づき適切に分配/合意形成を行う装置。システム全体として1つの状態を保つため
  \item 合意形成後のCommandを元にデータを変更する装置。システムの状態を更新するため
  \item Peerの管理を行い、システム全体の安定を図る。システムを停止させず、拡張や入れ替えを容易に行うため  
\end{enumerate}

智慧喰システムの準Peerは以下の装置によって構成されている。
\begin{enumerate}
  \item Command/Queryを受け取り、受け取った事実をトリガーとして任意のイベントを発生させる装置。受け取った事実を記録するため
  \item 受け取ったCommand/Queryの受容可能かを判断する装置。正しいProtocolに基づいた物のみを選別するため
  \item 合意形成後のCommandを元にデータを変更する装置。システムの状態を更新するため
  \item Peerの管理を行い、システム全体の安定を図る。システムを停止させず、拡張や入れ替えを容易に行うため  
\end{enumerate}

\subsubsection{プロトコル}


\subsection{特徴的な機能}

\subsubsection{三権分立的権限管理}
これからうめるよ〜

\subsubsection{移管}
これからうめるよ〜

\subsection{仕組み}

\subsubsection{台帳内でのデータ共有}
これからうめるよ〜

\subsubsection{台帳同士の結合}
これからうめるよ〜

\subsubsection{未来}
これからうめるよ〜

\end{multicols}
\end{document}
