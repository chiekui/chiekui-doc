\documentclass[a4j]{jarticle}
\usepackage{amsmath}
\usepackage[]{multicol}
\title{智慧喰 白書(draft)}
\author{水樹素子 sonoko\_mizuki}
\begin{document}
\maketitle
\begin{abstract}
この世には、様々な分散台帳システムが数多に存在しているが、未だ研究の余地があり課題を抱えている。
よってここに、人類を一歩すすめるため新しい分散台帳システム"智慧喰"を提案する。
この文書は、"智慧喰"の思想、設計、プロトコル、そして未来についてまとめたものである。
\end{abstract}

\begin{multicols}{2}
\newpage

\subsection{Introduction}
\subsubsection{由来と目的}
このシステムの名前を”智慧喰”としたのは以下の目的を満たすことが最終的な到達点だからである。
\begin{itemize}
  \item 智慧を食べるものとして知識を持った人々が集い、叡智を組み合わせより強固なシステムと成り、人類の礎を築く事を目的とする。
  \item 情報を食べるものとして異なる台帳が集い、巨大な1つの台帳と成り、利用者は情報の位置を知らなくとも参照、変更可能なシステムを築く事を目的とする。
\end{itemize}

\subsubsection{なせ作るか?}

\subsection{智慧喰い}
ここで私の提案するシステム、”智慧喰”について説明する。
\subsubsection{定義}

これからうめるよ〜

\subsubsection{構成機能}
これからうめるよ〜

\subsubsection{プロトコル}
これからうめるよ〜


\subsection{特徴的な機能}

\subsubsection{三権分立的権限管理}
これからうめるよ〜

\subsubsection{移管}
これからうめるよ〜

\subsection{仕組み}

\subsubsection{台帳内でのデータ共有}
これからうめるよ〜

\subsubsection{台帳同士の結合}
これからうめるよ〜

\subsubsection{未来}
これからうめるよ〜

\end{multicols}
\end{document}
